\section{Conclusiones Generales}
En general, todas las mediciones pudieron detectar al menos un cable intercontinental. Generalmente aquellos transoceánicos con grandes latencias. La fiabilidad de las rutas la pudimos contrastar con la implementación de Unix para traceroutes (implementada sobre UDP) y corroboramos que los resultados eran similares.

Sin embargo, pudimos notar muchas anomalías de diversos tipos.
\\

Una de las primeras anomalías con las que nos encontramos fue que aparecían IPs de España en el tramo Argentina-Miami en la medición de Nueva Zelanda, lo cual suponemos que esté relacionado a una asignación de IPs particular del ISP. Pero esto significó, desde la primer medición, entender la falibilidad natural de cualquier herramienta de geolocalización de IPs respecto a nuestras expectativas de que el entramado final siempre sea consistente. Lo cual se profundizó a medida que fuimos encontrando nodos geolocalizados en un extremo de enlace intercontinental, pero con latencias propias del otro extremo.
\\

Otra anomalía interesante fue que a veces los hosts destino no respondieron \emph{echo-reply} al recibir el mensaje o nodos intermedios que no responden los \emph{time-exceeded}. Comunmente, esto pasa porque hay un \emph{firewall} configurado para bloquear cierto tipo de comportamientos, en este caso de \emph{ICMP}. Usualmente es tanto para evitar \emph{smurf attacks (ICMP spoofing)\footnote{https://en.wikipedia.org/wiki/Smurf_attack}} como por motivos de performance. Lo cual quizás sea inesperado viniendo de universidades y no de empresas comerciales.
\\

Estuvieron presentes casos de falsos negativos y falsos positivos. Después de todo, la detección de enlaces de alta latencia como outliers se trata de una técnica probabilística y puede fallar. Razones simples para que esto suceda pueden ser que existan nodos que no conforman ningún extremo intercontinental en sobrecarga (como en el caso de la medición de Londres) o predominio de rutas de larga distancia con minoría de entramado local (como por ejemplo en la de Uzbekistán) que haga que la media de los ZRTT sea alta, y por lo tanto algunos cables de alta latencia no destaquen (en el ejemplo, Argentina-Miami) y algunos cables locales tengan un alto ZRTT por su bajo RTT respecto de la media inflada.

Otro motivo de falla, más natural y esperable, es el caso de la medición de Nueva Zelanda, donde el cable submarino entre Australia y Nueva Zelanda no presenta diferencias de RTT que ameriten un outlier.
